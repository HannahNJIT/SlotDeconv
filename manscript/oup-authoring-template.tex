%%
%% Copyright 2022 OXFORD UNIVERSITY PRESS
%%
%% This file is part of the 'oup-authoring-template Bundle'.
%% ---------------------------------------------
%%
%% It may be distributed under the conditions of the LaTeX Project Public
%% License, either version 1.2 of this license or (at your option) any
%% later version.  The latest version of this license is in
%%    http://www.latex-project.org/lppl.txt
%% and version 1.2 or later is part of all distributions of LaTeX
%% version 1999/12/01 or later.
%%
%% The list of all files belonging to the 'oup-authoring-template Bundle' is
%% given in the file `manifest.txt'.
%%
%% Template article for OXFORD UNIVERSITY PRESS's document class `oup-authoring-template'
%% with bibliographic references
%%

%%%CONTEMPORARY%%%
\documentclass[unnumsec,webpdf,contemporary,large]{oup-authoring-template}%
%\documentclass[unnumsec,webpdf,contemporary,large,namedate]{oup-authoring-template}% uncomment this line for author year citations and comment the above
%\documentclass[unnumsec,webpdf,contemporary,medium]{oup-authoring-template}
%\documentclass[unnumsec,webpdf,contemporary,small]{oup-authoring-template}

%%%MODERN%%%
%\documentclass[unnumsec,webpdf,modern,large]{oup-authoring-template}
%\documentclass[unnumsec,webpdf,modern,large,namedate]{oup-authoring-template}% uncomment this line for author year citations and comment the above
%\documentclass[unnumsec,webpdf,modern,medium]{oup-authoring-template}
%\documentclass[unnumsec,webpdf,modern,small]{oup-authoring-template}

%%%TRADITIONAL%%%
%\documentclass[unnumsec,webpdf,traditional,large]{oup-authoring-template}
%\documentclass[unnumsec,webpdf,traditional,large,namedate]{oup-authoring-template}% uncomment this line for author year citations and comment the above
%\documentclass[unnumsec,namedate,webpdf,traditional,medium]{oup-authoring-template}
%\documentclass[namedate,webpdf,traditional,small]{oup-authoring-template}

%\onecolumn % for one column layouts

%\usepackage{showframe}
\usepackage[utf8]{inputenc}
\usepackage{graphicx}
\usepackage{amsmath}
\usepackage{amssymb}
% \usepackage{geometry}
\usepackage{natbib}
% \geometry{left=2.5cm,right=2.5cm,top=2.5cm,bottom=2.5cm}
\usepackage{setspace}
\usepackage{titlesec}
\usepackage{authblk}

\graphicspath{{Fig/}}

% line numbers
%\usepackage[mathlines, switch]{lineno}
%\usepackage[right]{lineno}

\theoremstyle{thmstyleone}%
\newtheorem{theorem}{Theorem}%  meant for continuous numbers
%%\newtheorem{theorem}{Theorem}[section]% meant for sectionwise numbers
%% optional argument [theorem] produces theorem numbering sequence instead of independent numbers for Proposition
\newtheorem{proposition}[theorem]{Proposition}%
%%\newtheorem{proposition}{Proposition}% to get separate numbers for theorem and proposition etc.
\theoremstyle{thmstyletwo}%
\newtheorem{example}{Example}%
\newtheorem{remark}{Remark}%
\theoremstyle{thmstylethree}%
\newtheorem{definition}{Definition}

\begin{document}

\journaltitle{Journal Title Here}
\DOI{DOI HERE}
\copyrightyear{2022}
\pubyear{2019}
\access{Advance Access Publication Date: Day Month Year}
\appnotes{Paper}

\firstpage{1}

%\subtitle{Subject Section}

\title[SlotDeconv]{\Large\bfseries SlotDeconv: Spatial Transcriptomics Deconvolution via Diversity-Constrained Prototype Learning and Spatial Refinement}


% \title{\Large\bfseries GROVE: Generative Spatial Deconvolution via Disentangled Latent Factors and Domain-Adaptive Decoder Transfer}
% \author[1,*]{Hanzhang Fang}
% \author[2]{Yuanjia Zou}
% \author[2]{Yeqing Chen}
% \author[3]{Cong Qi}
% \author[1]{Zhi Wei}
% \affil[1]{Department of Computer Science, New Jersey Institute of Technology, Newark, NJ, USA}
% \affil[*]{Corresponding author. E-mail: zhi.wei@njit.edu}

% \author[1]{Hanzhang Fang}
% \author[2]{Yuanjie Zou}
% \author[3]{Yeqing Chen}
% \author[3]{Cong Qi}
% \author[4]{Zhi Wei\ORCID{0000-0000-0000-0000}}

\author{
Hanzhang Fang$^{1}$,
Cong Qi$^{1}$,
Yuanjie Zou$^{1}$,
Yeqing Chen$^{1}$,
Zhi Wei$^{1, \ast}$\ORCID{0000-0000-0000-0000}
}

\authormark{Hanzhang et al.}

\address[1]{\orgdiv{Department of Computer Science}, \orgname{New Jersey Institute of Technology}, \orgaddress{\street{University Heights}, \postcode{07102}, \state{New Jersey}, \country{USA}}}
% \address[2]{\orgdiv{Department}, \orgname{Organization}, \orgaddress{\street{Street}, \postcode{Postcode}, \state{State}, \country{Country}}}
% \address[3]{\orgdiv{Department}, \orgname{Organization}, \orgaddress{\street{Street}, \postcode{Postcode}, \state{State}, \country{Country}}}
% \address[4]{\orgdiv{Department}, \orgname{Organization}, \orgaddress{\street{Street}, \postcode{Postcode}, \state{State}, \country{Country}}}

\corresp[$\ast$]{Corresponding author. \href{zhi.wei@njit.edu}{zhi.wei@njit.edu}}

\received{Date}{0}{Year}
\revised{Date}{0}{Year}
\accepted{Date}{0}{Year}

%\editor{Associate Editor: Name}

%\abstract{
%\textbf{Motivation:} .\\
%\textbf{Results:} .\\
%\textbf{Availability:} .\\
%\textbf{Contact:} \href{name@email.com}{name@email.com}\\
%\textbf{Supplementary information:} Supplementary data are available at \textit{Journal Name}
%online.}

\abstract{\textbf{Motivation:} Spatial transcriptomics enables spatially resolved gene expression profiling, yet sequencing-based platforms capture multi-cell mixtures per spot, necessitating computational deconvolution. Existing approaches often rely on fixed reference signatures or learn them without explicit separability constraints, yielding ill-conditioned linear systems when closely related subtypes exhibit highly similar expression profiles. Moreover, many methods treat spots independently, ignoring the spatial autocorrelation inherent to tissue architecture. \\ \textbf{Results:} We present SlotDeconv, a three-stage framework that decouples nonlinear signature learning from linear mixture inference for robust proportion estimation. Stage~1 learns discriminative cell-type signatures via learnable prototype vectors (``slots'') decoded through a neural network under a negative binomial likelihood. A max-margin diversity loss explicitly penalizes inter-signature cosine similarity, improving separability even among fine-grained subtypes within the same lineage. Stage~2 initializes proportions via discriminative-gene-weighted nonnegative least squares (NNLS), leveraging the stabilized linear system induced by the learned signatures. Stage~3 refines proportions by minimizing the KL divergence between empirical per-spot gene distributions and normalized reconstructions, together with Gaussian-kernel neighborhood consistency on the proportion matrix. On a mouse brain ST dataset with 27 cell types, SlotDeconv achieves a spot-wise Pearson correlation of 0.55, a 29\% relative improvement over baseline NNLS (0.43), with particularly strong gains in resolving highly similar cortical-layer neuronal subtypes. We use ``slots'' to denote learnable prototypes; unlike Slot Attention, we do not employ iterative attention-based binding. \\ \textbf{Availability:} Source code is available at \url{https://github.com/xxx/SlotDeconv} {\textcolor{red}{ less than 250 words}} }


% \boxedtext{
% \begin{itemize}
% \item Key boxed text here.
% \item Key boxed text here.
% \item Key boxed text here.
% \end{itemize}}

\maketitle
% 1. 导入摘要
% \begin{abstract}
\textbf{Motivation:} Spatial transcriptomics enables spatially resolved gene expression profiling, yet sequencing-based platforms capture multi-cell mixtures per spot, necessitating computational deconvolution. Existing approaches often rely on fixed reference signatures or learn them without explicit separability constraints, yielding ill-conditioned linear systems when closely related subtypes exhibit highly similar expression profiles. Moreover, many methods treat spots independently, ignoring the spatial autocorrelation inherent to tissue architecture.

\textbf{Results:} We present SlotDeconv, a three-stage framework that decouples nonlinear signature learning from linear mixture inference for robust proportion estimation. Stage~1 learns discriminative cell-type signatures via learnable prototype vectors (``slots'') decoded through a neural network under a negative binomial likelihood. A max-margin diversity loss explicitly penalizes inter-signature cosine similarity, improving separability even among fine-grained subtypes within the same lineage. Stage~2 initializes proportions via discriminative-gene-weighted nonnegative least squares (NNLS), leveraging the stabilized linear system induced by the learned signatures. Stage~3 refines proportions by minimizing the KL divergence between empirical per-spot gene distributions and normalized reconstructions, together with Gaussian-kernel neighborhood consistency on the proportion matrix. On a mouse brain ST dataset with 27 cell types, SlotDeconv achieves a spot-wise Pearson correlation of 0.55, a 29\% relative improvement over baseline NNLS (0.43), with particularly strong gains in resolving highly similar cortical-layer neuronal subtypes. We use ``slots'' to denote learnable prototypes; unlike Slot Attention, we do not employ iterative attention-based binding.

\textbf{Availability:} Source code is available at \url{https://github.com/xxx/SlotDeconv}.
\end{abstract}
% 2. 导入引言
\section{Introduction}
Spatial transcriptomics (ST) enables gene expression profiling while preserving spatial context, but sequencing-based platforms such as 10x Visium and Slide-seq operate at spot-level resolution where each location aggregates transcripts from multiple cells~\cite{staahl2016visualization}. Cell-type deconvolution---inferring constituent cell-type proportions within each spot---has therefore become essential for downstream spatial analyses~\cite{gaspard2025cell}.


Existing deconvolution methods can be categorized along two dimensions: modeling approach (linear vs.\ nonlinear) and spatial awareness~\cite{gaspard2025cell,sang2024spotless}.
\textbf{Linear models without spatial information} dominate the field. Probabilistic methods include RCTD~\cite{cable2022robust} (likelihood-based with platform normalization), Cell2location~\cite{kleshchevnikov2022cell2location} (hierarchical Bayesian with negative binomial likelihood), Stereoscope~\cite{andersson2020single}, and DestVI~\cite{lopez2022destvi} (variational inference for continuous cell states). Matrix factorization approaches include SPOTlight~\cite{elosua2021spotlight} (NMF+NNLS) and SpatialDWLS~\cite{dong2021spatialdwls}. Optimal transport methods such as Tangram~\cite{biancalani2021deep} learn probabilistic cell-to-spot mappings.
\textbf{Linear models with spatial information} leverage tissue continuity. CARD~\cite{ma2022spatially} pioneered conditional autoregressive modeling to borrow information across neighbors. DeCoST~\cite{guo2025decost} extends this with Gaussian kernel-based CAR and domain adaptation for platform effects. SONAR~\cite{liu2023sonar} uses spatially weighted Poisson-Gamma models.
\textbf{Nonlinear deep learning approaches} capture complex relationships beyond linear mixtures. GraphST~\cite{long2023spatially} combines graph neural networks with contrastive learning for joint spatial clustering and deconvolution. Spotiphy~\cite{yang2025spotiphy} uses generative modeling to achieve pseudo-single-cell-resolution whole-transcriptome imaging.


Despite progress, existing methods face two fundamental challenges:
\textbf{(1) Reference signature quality.} Cluster-wise averaging from scRNA-seq produces highly collinear signatures for related subtypes (e.g., cortical layers L2/3, L4, L5), leading to ill-conditioned linear systems~\cite{cable2022robust,gaspard2025cell}.
\textbf{(2) Spatial smoothing trade-offs.} Methods incorporating spatial priors risk oversmoothing anatomical boundaries or provide insufficient denoising~\cite{ma2022spatially}.
To address these challenges, we propose SlotDeconv, occupying a middle ground between linear methods and end-to-end deep models: \textbf{nonlinear representation learning} for discriminative signatures combined with \textbf{linear mixture deconvolution} for interpretability.
Our three-stage framework:
\textbf{Stage 1: Constrained Reference Learning.} Each cell type is represented by a learnable prototype (``slot''), mapped to gene space through a neural decoder trained under negative binomial likelihood~\cite{love2014moderated}. A max-margin diversity loss penalizes excessive cosine similarity between signatures, improving discriminability for closely related subtypes. Unlike Slot Attention in object-centric vision, we do not employ iterative attention-based binding.
\textbf{Stage 2: Warm-Start Initialization.} Discriminative-gene-weighted NNLS (with gene-level F-statistics) provides a fast and stable initialization under a convex formulation.
\textbf{Stage 3: Spatial Refinement.} Starting from the NNLS solution, proportions are refined by minimizing KL divergence between empirical per-spot gene distributions and normalized reconstructions, together with a Gaussian-kernel neighborhood smoothness prior on the proportion matrix. The kernel bandwidth is set using the median inter-spot distance, reducing sensitivity to coordinate scaling.
\textbf{Contributions:} We (1) learn subtype-discriminative signatures via diversity-regularized prototype decoding under negative binomial modeling; (2) combine weighted NNLS warm-start with KL-based spatial refinement using a median-distance-scaled neighborhood kernel; and (3) achieve a 29\% relative improvement over NNLS on a mouse brain ST dataset with 27 cell types, with improved cortical-layer subtype resolution.

% 3. 材料与方法部分 (Materials and methods)
\section*{Materials and methods}
    % 导入 Benchmark datasets 文件里第一行是 \subsection*{Benchmark datasets})
    \input{sections/Benchmark_datasets}
    % 导入 Methods (注意:确保该 .tex 文件里第一行是 \subsection*{Methods})
    \section{Methods}
\subsection{Overview}
SlotDeconv is a three-stage framework for spatial transcriptomics (ST) deconvolution. Given scRNA-seq counts with cell-type annotations, ST counts, and spot coordinates, SlotDeconv (i) learns a discriminative cell-type signature matrix, (ii) obtains a convex warm start by weighted non-negative least squares (NNLS), and (iii) refines spot-wise proportions with a spatial neighborhood consistency prior.
Let $\mathbf{Y}\in\mathbb{R}_+^{C\times G_0}$ denote the scRNA-seq count matrix with $C$ cells and $G_0$ genes, and let $\boldsymbol{\tau}\in\{1,\ldots,K\}^C$ denote cell-type labels over $K$ cell types. Let $\mathbf{X}\in\mathbb{R}_+^{N\times G_0}$ denote the ST count matrix with $N$ spots, and let $\mathbf{C}\in\mathbb{R}^{N\times 2}$ denote spot coordinates. The output is the proportion matrix $\mathbf{V}\in\mathbb{R}_+^{N\times K}$, where each row lies on the simplex: $\sum_{k=1}^K V_{ik}=1$ and $V_{ik}\ge 0$.
Stage~1 learns a nonnegative, row-normalized signature matrix $\mathbf{B}\in\mathbb{R}_+^{K\times G}$ on a selected gene set of size $G$. Stage~2 computes an initialization $\mathbf{V}^{(\mathrm{init})}$ by weighted NNLS on library-size normalized ST expression. Stage~3 refines $\mathbf{V}$ by minimizing a KL reconstruction term plus a dense Gaussian-kernel spatial smoothness term.

\begin{table}[t]
\small
\caption{Notation summary.}
\label{tab:notation}
\centering
\begin{tabular}{cl}
\toprule
\textbf{Symbol} & \textbf{Description} \\
\midrule
$\mathbf{Y}\in\mathbb{R}_+^{C\times G_0}$ & scRNA-seq count matrix (cells $\times$ genes) \\
$\boldsymbol{\tau}\in\{1,\ldots,K\}^C$ & scRNA-seq cell-type labels \\
$\mathbf{X}\in\mathbb{R}_+^{N\times G_0}$ & ST count matrix (spots $\times$ genes) \\
$\mathbf{C}\in\mathbb{R}^{N\times 2}$ & spot coordinates \\
$\mathbf{S}\in\mathbb{R}^{K\times d}$ & learnable prototype (slot) vectors \\
$f_\theta$ & decoder network with parameters $\theta$ \\
$\tilde{\mathbf{b}}_k\in\mathbb{R}_+^{G}$ & unnormalized signature for cell type $k$ \\
$\mathbf{B}\in\mathbb{R}_+^{K\times G}$ & row-normalized signature matrix ($\sum_g B_{kg}=1$) \\
$\boldsymbol{\alpha}\in\mathbb{R}_+^{G}$ & gene-wise NB dispersion (shape) parameters \\
$\mathbf{w},\tilde{\mathbf{w}}\in\mathbb{R}_+^{G}$ & raw and transformed gene weights for NNLS \\
$\mathbf{V}\in\mathbb{R}_+^{N\times K}$ & cell-type proportions (rows on simplex) \\
$\mathbf{L}\in\mathbb{R}^{N\times K}$ & unconstrained logits for Stage~3 optimization \\
$\bar{\mathbf{A}}\in\mathbb{R}_+^{N\times N}$ & row-normalized dense spatial kernel with zero diagonal \\
\bottomrule
\end{tabular}
\end{table}

\subsection{Data preprocessing}
\textbf{Cell balancing (scRNA-seq).} To reduce dominance of abundant cell types, we subsample each cell type to at most $M$ cells (default $M=750$), yielding a balanced scRNA-seq set.
\textbf{Gene selection and discriminative scores.} On the balanced scRNA-seq data, we compute gene-wise discriminative scores $\mathbf{w}\in\mathbb{R}_+^{G_0}$ using an ANOVA-like ratio of between-type to within-type variance. We select the top $G$ genes (default $G=3000$) and use the intersection of selected genes with genes present in the ST matrix for all subsequent stages.
\textbf{Median-normalized size factors.} For scRNA-seq cell $c$, define $s_c=\sum_g Y_{cg}$ and
\begin{equation}
\tilde{s}_c=\frac{s_c}{\mathrm{median}(\{s_{c'}:s_{c'}>0\})}.
\end{equation}
For ST spot $i$, define $l_i=\sum_g X_{ig}$ and
\begin{equation}
\tilde{l}_i=\frac{l_i}{\mathrm{median}(\{l_{i'}:l_{i'}>0\})}.
\end{equation}
These size factors are treated as \textbf{constants} in subsequent optimization.
\textbf{Transformed gene weights for NNLS.} To stabilize weighted NNLS, we apply power scaling and clipping (default $\gamma=0.8$):
\begin{equation}
\tilde{w}_g=\mathrm{clip}\!\left(\left(\frac{w_g}{\mathrm{median}(\mathbf{w})+\epsilon}\right)^{\gamma},\,0.2,\,5.0\right),
\end{equation}
and use $\sqrt{\tilde{w}_g}$ as diagonal weights in Stage~2.

\subsection{Model architecture and implementation}
\subsubsection{Stage 1: Constrained reference learning}
\textbf{Constants, parameters.} Constants are $\boldsymbol{\tau}$ and $\{\tilde{s}_c\}_{c=1}^C$. Trainable parameters are prototype vectors $\mathbf{S}\in\mathbb{R}^{K\times d}$ (default $d=128$), decoder parameters $\theta$, and gene-wise dispersion parameters $\boldsymbol{\alpha}\in\mathbb{R}_+^{G}$.
\textbf{Decoder architecture and forward computation.} The decoder is a 3-layer MLP with widths $d\!\to\!256\!\to\!512\!\to\!G$. It applies LayerNorm and ReLU after the first two linear layers, with a dropout rate of 0.1 applied after the first LayerNorm+ReLU block during training. For cell type $k$, we compute logits $\mathbf{h}_k=f_\theta(\mathbf{s}_k)$ and obtain a nonnegative unnormalized signature
\begin{equation}
\tilde{\mathbf{b}}_k=\mathrm{softplus}\!\big(\mathrm{clip}(\mathbf{h}_k,-20,20)\big)\in\mathbb{R}_+^{G}.
\end{equation}
We then define the row-normalized signature matrix used in Stage~2--3 and in the diversity term:
\begin{equation}
B_{kg}=\frac{\tilde{b}_{kg}}{\sum_{g'}\tilde{b}_{kg'}+\epsilon},\quad \mathbf{B}\in\mathbb{R}_+^{K\times G},\quad \sum_g B_{kg}=1.
\end{equation}
\textbf{Negative binomial likelihood.} For scRNA-seq cell $c$ with type $\tau(c)$, we model counts with mean $\mu_{cg}=\tilde{s}_c\cdot \tilde{b}_{\tau(c),g}$ and dispersion (shape) $\alpha_g$, using a mean--shape parameterization with $\mathrm{Var}(y_{cg})=\mu_{cg}+\mu_{cg}^2/\alpha_g$:
\begin{equation}
\mathcal{L}_{\mathrm{NB}}=-\frac{1}{CG}\sum_{c=1}^{C}\sum_{g=1}^{G}\log \mathrm{NB}(y_{cg};\mu_{cg},\alpha_g).
\end{equation}
\textbf{Worst-case max-margin diversity loss.} Let $\hat{\mathbf{B}}_k=\mathbf{B}_k/\|\mathbf{B}_k\|_2$ and $S_{ij}=\hat{\mathbf{B}}_i^\top \hat{\mathbf{B}}_j$. We penalize the most violating off-diagonal pair:
\begin{equation}
\mathcal{L}_{\mathrm{div}}=\max_{i\neq j}\big[ S_{ij}-m\big]_+,
\end{equation}
where $m$ is a margin (default $m=0.1$).
\textbf{Stage~1 objective.} We minimize
\begin{equation}
\mathcal{L}_1=\mathcal{L}_{\mathrm{NB}}+\lambda_{\mathrm{div}}\mathcal{L}_{\mathrm{div}},
\end{equation}
and optimize $\{\mathbf{S},\theta,\boldsymbol{\alpha}\}$ using AdamW with cosine annealing and gradient clipping (max norm 5.0). After convergence, we fix the learned $\mathbf{B}$ for Stage~2--3.

\subsubsection{Stage 2: Warm-start initialization (weighted NNLS)}
\textbf{Constants, variables.} Constants are $\mathbf{B}$, transformed gene weights $\tilde{\mathbf{w}}$, and ST size factors $\{\tilde{l}_i\}_{i=1}^{N}$. Variables are spot-wise proportions $\mathbf{v}_i^{(0)}\in\mathbb{R}_+^{K}$.
\textbf{Computation order.} We compute library-size normalized ST expression $\mathbf{x}_i^{(u)}=\mathbf{x}_i/\tilde{l}_i$ and solve weighted NNLS:
\begin{equation}
\mathbf{v}_i^{(0)}=\arg\min_{\mathbf{v}\ge 0}\left\|\mathrm{diag}(\tilde{\mathbf{w}}^{1/2})\left(\mathbf{x}_i^{(u)}-\mathbf{B}^\top\mathbf{v}\right)\right\|_2^2.
\end{equation}
We project onto the simplex by $\ell_1$ normalization:
\begin{equation}
\mathbf{v}_i^{(\mathrm{init})}=\frac{\mathbf{v}_i^{(0)}}{\sum_k v_{ik}^{(0)}}\ \text{if }\sum_k v_{ik}^{(0)}>0,\quad \text{else }\ \mathbf{v}_i^{(\mathrm{init})}=\mathbf{1}_K/K.
\end{equation}

\subsubsection{Stage 3: Spatial-aware refinement (logit parameterization)}
\textbf{Constants, variables.} Constants are $\mathbf{B}$ and the observed per-spot gene distributions $\mathbf{X}_{\mathrm{prob}}$. Variables are logits $\mathbf{L}\in\mathbb{R}^{N\times K}$ with row-wise softmax $\mathbf{V}=\mathrm{softmax}(\mathbf{L})$. Initialization is $\mathbf{L}^{(0)}=\log(\mathbf{V}^{(\mathrm{init})}+\epsilon)$.
\textbf{KL reconstruction term.} Define empirical gene distributions $X^{\mathrm{prob}}_{ig}=X_{ig}/(\sum_{g'}X_{ig'}+\epsilon)$ and predicted distributions from $\hat{\mathbf{x}}_i=\mathbf{v}_i\mathbf{B}$:
\begin{equation}
\hat{x}^{\mathrm{prob}}_{ig}=\frac{\hat{x}_{ig}}{\sum_{g'}\hat{x}_{ig'}+\epsilon}.
\end{equation}
We minimize $\mathrm{KL}(X_{\mathrm{prob}}\|\hat{X}_{\mathrm{prob}})$ up to an additive constant:
\begin{equation}
\mathcal{L}_{\mathrm{KL}}=\frac{1}{N}\sum_{i=1}^{N}\sum_{g=1}^{G}X^{\mathrm{prob}}_{ig}\left(\log(X^{\mathrm{prob}}_{ig}+\epsilon)-\log(\hat{x}^{\mathrm{prob}}_{ig}+\epsilon)\right).
\end{equation}
\textbf{Dense Gaussian-kernel neighborhood consistency.} We z-score normalize coordinates $\tilde{\mathbf{c}}_i=(\mathbf{c}_i-\bar{\mathbf{c}})/(\boldsymbol{\sigma}_{\mathbf{c}}+\epsilon)$ and compute pairwise distances $d_{ij}=\|\tilde{\mathbf{c}}_i-\tilde{\mathbf{c}}_j\|_2$. The bandwidth is set to $\sigma=\mathrm{median}(\{d_{ij}:d_{ij}>0\})$. We define a dense Gaussian kernel, remove self-loops, and row-normalize:
\begin{equation}
\tilde{A}_{ij}=\exp\!\left(-\frac{d_{ij}^2}{2\sigma^2}\right)(1-\delta_{ij}),\quad \bar{A}_{ij}=\frac{\tilde{A}_{ij}}{\sum_{j'}\tilde{A}_{ij'}+\epsilon}.
\end{equation}
The spatial smoothness loss is implemented as the mean squared deviation from neighborhood-averaged proportions:
\begin{equation}
\mathcal{L}_{\mathrm{sp}}=\frac{1}{NK}\sum_{i=1}^{N}\sum_{k=1}^{K}\left(V_{ik}-(\bar{\mathbf{A}}\mathbf{V})_{ik}\right)^2.
\end{equation}
\textbf{Stage~3 objective.} We minimize
\begin{equation}
\mathcal{L}_3=\mathcal{L}_{\mathrm{KL}}+\lambda_{\mathrm{sp}}\mathcal{L}_{\mathrm{sp}},
\end{equation}
and optimize $\mathbf{L}$ using Adam with cosine annealing.

\subsection{Implementation details}
SlotDeconv is implemented in Python using PyTorch and SciPy. Stage~2 NNLS is solved by \texttt{scipy.optimize.nnls}.
Default hyperparameters are $G=3000$, $M=750$, $d=128$, $\lambda_{\mathrm{div}}=4.0$, $m=0.1$, $\gamma=0.8$, and $\lambda_{\mathrm{sp}}=15.0$.
Stage~1 uses AdamW with learning rate $10^{-3}$, weight decay $10^{-5}$, and runs for 2000 epochs.
Stage~3 uses Adam with learning rate $10^{-2}$ and runs for 1500 epochs.
Both stages employ cosine annealing learning rate scheduling.
% 4. 其他部分 (可选)
% \input{sections/Results}
% \input{sections/Discussion}

% 5. 参考文献
\bibliographystyle{unsrtnat} % 建议把样式放在这里,紧挨着 bib 文件
\bibliography{cite}         % 注意:这里必须是 cite,对应你的 cite.bib 文件


\end{document}
